% Author: Rasmus Pank Roulund
% !TEX root = ./main.tex

%%% Setup Page Layout

\documentclass[
	paper=a4,
	fontsize=12pt,
	BCOR=5mm,
%	openany,
	headinclude,
	% fontinclude=true,
%	cleardoublepage=empty,
	pagesize=pdftex,
	headsepline,							% Line under page header
	%footsepline,
	%toc=bib, index=totoc,
	toc=listof,
	%report,
	toc=bibliography,
	% setcapindent,
	% titlepage=true,
	english
]{scrreprt}

\usepackage[onehalfspacing]{setspace}	% Should set up onehalfspacing, copied from Steffen
% \setlength{\parskip}{0.5cm plus0.2cm minus 0.2cm}

%%% Setup Encoding and Fonts

\usepackage[utf8]{inputenx}
\usepackage[T1]{fontenc}
\usepackage{lmodern}
% Somehow, lmodern has to be loaded before mathdesign. Otherwise, the font looks like garbage
\usepackage[bitstream-charter]{mathdesign}	% Charter-BT Font
%\let\circledS\undefined		% suppress error of multiply defined \circledS
% \usepackage{amssymb}
%\usepackage{mdbch}

%%% Setup Document Language

\usepackage[english]{babel}

%%% Setup Graphics

\usepackage{graphicx}
\usepackage{subcaption}
	\graphicspath{ {images/} }

% Used to outsource tikz code and include it in a figure environment
\usepackage{standalone}
\usepackage{tikz}
\usetikzlibrary{shapes,arrows}

%%% Setup Headers and Footers

\usepackage{scrlayer-scrpage}		% Part of KOMA-Script, formatting headers and footers
\clearscrheadfoot				% remove standard head- and footline
\automark*{chapter}				% redefine chapter style
\ohead*{\pagemark}				% Sets pagenumber on outer headline, including chapterpage
\ihead{\headmark}				% Sets current section on inner headline, excluding chapterpage

%%% Setup Tables

\usepackage{tabularx}					% https://www.ctan.org/pkg/tabularx page wide tables
\usepackage{multirow}					% https://www.ctan.org/pkg/multirow
\usepackage{booktabs}					% https://www.ctan.org/pkg/booktabs nicer tables
\usepackage{float}						% https://www.ctan.org/pkg/float
\usepackage{longtable}					% https://ctan.org/pkg/longtable tables spanning more than one page

\usepackage[
	justification=RaggedRight,
	singlelinecheck=false
]{caption}

%%% Setup Bibliography

\usepackage{csquotes}					% https://www.ctan.org/pkg/csquotes required for biblatex
\usepackage[							% https://www.ctan.org/pkg/biblatex
	style=chem-acs,						% American chemical society citation style
	autocite=superscript				% Sets numbers in superscript
]{biblatex}
\addbibresource{references/library.bib}	% Bibliography location

%%% Misc. Packages

\usepackage{siunitx}					% Formatting SI units
	\sisetup{detect-all}				% Changes SI-number font to global font
\usepackage{chemformula}				% Formatting chemical elements
\usepackage{microtype}					% khirevich.com/latex/microtype
\usepackage{pdfpages}					% Include titlepage as pdf
\usepackage{scrhack}					% Removes Errors from Koma-Script Packages
\usepackage{todonotes}					% Adds Todo-Notes
\usepackage{textgreek}					% Allow greek letters without math mode


%%% Custom Comands

% Shortcuts for species names
\newcommand{\coli}{\textit{E. coli}~K12}		
\newcommand{\bac}{\textit{B. subtilis}~168}
\newcommand{\tue}{\textit{Streptomyces}~sp.~Tü2401}

% Shortcuts for columns
\newcommand{\luna}{Luna\textsuperscript{\textregistered} NH\textsubscript{2}}

% Other Shortcuts
\newcommand{\pka}{p\textit{K}\textsubscript{a}}
\usepackage{tikz}
\usetikzlibrary{calc,trees,positioning,arrows,chains,shapes.geometric,%
    decorations.pathreplacing,decorations.pathmorphing,shapes,%
    matrix,shapes.symbols}

\tikzset{
>=stealth',
  punktchain/.style={
    rectangle, 
    rounded corners, 
    % fill=black!10,
    draw=black, very thick,
    text width=10em, 
    minimum height=3em, 
    text centered, 
    on chain},
  line/.style={draw, thick, <-},
  element/.style={
    tape,
    top color=white,
    bottom color=blue!50!black!60!,
    minimum width=8em,
    draw=blue!40!black!90, very thick,
    text width=10em, 
    minimum height=3.5em, 
    text centered, 
    on chain},
  every join/.style={->, thick,shorten >=1pt},
  decoration={brace},
  tuborg/.style={decorate},
  tubnode/.style={midway, right=2pt},
}
\begin{document}
\begin{tikzpicture}
  [node distance=.8cm,
  start chain=going below,]
%     \node[punktchain, join] (intro) {Introduktion};
%     \node[punktchain, join] (probf)      {Problemformulering};
%     \node[punktchain, join] (investeringer)      {Investeringsteori};
%     \node[punktchain, join] (perfekt) {Det perfekte kapitalmarked};
%     \node[punktchain, join, ] (emperi) {Emperi};
%      \node (asym) [punktchain ]  {Asymmetrisk information};
%      \begin{scope}[start branch=venstre,
%        %We need to redefine the join-style to have the -> turn out right
%        every join/.style={->, thick, shorten <=1pt}, ]
%        \node[punktchain, on chain=going left, join=by {<-}]
%            (risiko) {Risiko og gamble};
%      \end{scope}
%      \begin{scope}[start branch=hoejre,]
%      \node (finans) [punktchain, on chain=going right] {Det finansielle system};
%    \end{scope}
%  \node[punktchain, join,] (disk) {Det imperfekte finansielle marked};
%  \node[punktchain, join,] (makro) {Investeringsmæssige konsekvenser};
%  \node[punktchain, join] (konk) {Konklusion};
%  % Now that we have finished the main figure let us add some "after-drawings"
%  %% First, let us connect (finans) with (disk). We want it to have
%  %% square corners.
%  \draw[|-,-|,->, thick,] (finans.south) |-+(0,-1em)-| (disk.north);
%  % Now, let us add some braches. 
%  %% No. 1
%  \draw[tuborg] let
%    \p1=(risiko.west), \p2=(finans.east) in
%    ($(\x1,\y1+2.5em)$) -- ($(\x2,\y2+2.5em)$) node[above, midway]  {Teori};
%  %% No. 2
%  \draw[tuborg, decoration={brace}] let \p1=(disk.north), \p2=(makro.south) in
%    ($(2.5, \y1)$) -- ($(2.5, \y2)$) node[tubnode] {Analyse};
%  %% No. 3
%  \draw[tuborg, decoration={brace}] let \p1=(perfekt.north), \p2=(emperi.south) in
%    ($(2.5, \y1)$) -- ($(2.5, \y2)$) node[tubnode] {Problemfelt};

	\node[punktchain, join]	(2401)	{Tü2401 reverse extract};
	\node[punktchain, join]	(1)		{LC-MS};
  \end{tikzpicture}
\end{document}