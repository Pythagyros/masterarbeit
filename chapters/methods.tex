\section{Chemicals \& Instruments} % (fold)
\label{sec:chemicals_&_instruments}

Chemicals
% Acetonitrile - J. T. Baker
% Methyl acetate - Alfa Aesar

The used instruments are listed in in table~\ref{tab:labins}.

\begin{table}[H]
	\caption{Used laboratory Instruments}
	\label{tab:labins}
	\centering
	\begin{tabularx}{\textwidth}{XXX}
		\toprule
		\textbf{Instrument}			& \textbf{Model}		& \textbf{Manufacturer}	\\
		\midrule
		Centrifuges			&	Megafuge 1.0 (R)		&	Heraeus Instruments GmbH, Hanau, Germany	\\
							&	Centrifuge 5417 C 		&	Eppendorf AG, Hamburg, Germany				\\
							&	RC6 Plus Centrifuge 	&	Sorvall Ltd., Delaware, USA					\\
		Lyophilizator		&	LyoVac GT2				&	Leybold GmbH, Cologne, Germany				\\
		Spectrophotometer	&	BioMate 3S				&	Thermo Fisher, Waltham, USA					\\
		Rotary Evaporator	&	Hei-Vap Precision		&	Heidolph Instruments GmbH, Schwabach, Germany	\\
							&	Rotavapor RE + PC 3001 VARIO Pump	&	Büchi Labortechnik AG, Flawil, Switzerland + vacuubrand GmbH, Essen, Germany \\

		\bottomrule
	\end{tabularx}
\end{table}

High performance liquid chromatography (HPLC) systems were manufactured by AgilentThe components of the HPLC systems are listed in Table~\ref{tab:HPLCtab}.

\begin{table}[H]
	\caption{Components of HPLC systems}
	\label{tab:HPLCtab}
	\centering
	\begin{tabularx}{\textwidth}{XXX}
		\toprule
						& \textbf{Component}		& \textbf{Description}	\\
		\midrule
		Agilent 1100 Series		&	G1322A		&	Degasser			\\
								&	G1311A		&	Quaternary Pump		\\
								&	G1313A		&	Autosampler			\\
								&	G1316A		&	Column Compartment	\\
								&	G1315B		&	Diode Array Detector	\\
		Agilent 1200 Series		&	G1379B		&	Degasser			\\
								&	G1312A		&	Binary Pump			\\
								&	G1367B		&	Autosampler			\\
								&	G1330B		&	Thermostat			\\
								&	G1316A		&	Column Compartment	\\
								&	G1315B		&	Diode Array Detector	\\
		Agilent 1260 Infinity	&	G4225A		&	Degasser			\\
								&	G1312C		&	Binary Pump			\\
								&	G1329B		&	Autosampler			\\
								&	G1330B		&	Thermostat			\\
								&	G1316A		&	Column Compartment	\\
								&	G1315D		&	Diode Array Detector	\\


		\bottomrule
	\end{tabularx}
\end{table}

% section chemicals_&_instruments (end)

\section{Strain Cultivation} % (fold)
\label{sec:strain_cultivation}

	\subsection{Batch Fermentation} % (fold)
	\label{sub:fermentation}
	The strain Tü2401 was cultivated at a ten-litre scale in a continuous stirred tank bioreactor. 500 ml of pre-culture were grown in five 500 ml round flasks containing 100 ml of NL 410 medium without \ch{CaCO3}. The pre-cultures were inoculated from stored ISP-agar plates and grown for 72 h at 27 °C. The pre-cultures were pooled and used to inoculate 9.5 l of NL OM medium for fermentation. The temperature was kept at 27 °C with an airflow of 5 l/min and a rotor speed of 200 rpm. Control samples of 15 ml were taken throughout the process at regular intervals. Fermentation was stopped after 125 h and the culture broth was harvested. Further processing is described in \ref{sub:processing_of_fermentation_broth}.
	
	% subsection fermentation (end)
% section strain_cultivation (end)

\section{Sample Preparation} % (fold)
\label{sec:sample_preparation}
% section sample_preparation (end)

	\subsection{Processing of Fermentation Broth} % (fold)
	\label{sub:processing_of_fermentation_broth}
	The harvested fermentation broth was supplemented with diatomaceous earth and filtered through Pall T 1500 filter plates (relative retention range 10 - 30 µm). The remaining filter cake was discarded and the filtrate transferred to a stirring bucket % englischen begriff nachsehen
	Two liters of ethyl acetate were added to the filtrate and stirred for 30 min. After completed phase-separation, the organic phase was collected and the aqueous phase reused for further extraction. The process was repeated five times.
	% subsection processing_of_fermentation_broth (end)

\section{Chromatographic Methods} % (fold)
\label{sec:chromatographic_methods}
% All Temperatures set at 25 °C
% section chromatographic_methods (end)

	\subsection{Thin Layer Chromatography} % (fold)
	\label{sub:thin_layer_chromatography}
	% subsection thin_layer_chromatography (end)

	\subsection{Ion Exchange Chromatography} % (fold)
	\label{sub:ion_exchange_chromatography}
	% subsection ion_exchange_chromatography (end)

	\subsection{Hydrophilic Interaction Chromatography} % (fold)
	\label{sub:hilic}

	Hydrophilic Interaction Chromatography (HILIC) with performed with a 4,6 x 250 mm ZIC-HILIC Column (Merck). It features zwitterionic, functional groups on poly(etherether ketone) (PEEK) material.
	10 mM Ammonium acetate in Milli-Q \ch{H2O} was used as solvent A, while Acetonitrile comprised solvent B. Detailed method descriptions regarding solvent composition, flow and duration are listed in the appendix.
	% subsection hilic (end)

	\subsection{High Performance Liquid Chromatography} % (fold)
	\label{sub:hplc}
	% subsection hplc (end)

	\subsection{Mass Spectrometry} % (fold)
	\label{sub:mass_spectrometry}

A test table should be here

\begin{table}[h]
	\caption{A test table for HPLC Methods}
	\label{tab:asddf}
	\centering
	\begin{tabularx}{\textwidth}{XXX}
		%\toprule
						& \textbf{Component}		& \textbf{Description}	\\
		\midrule
		HPLC Parameters & System			& 	\\
						& Column			& 	\\
						& Injection volume 	& \SI{50}{\micro\liter}	\\
						& Flow				& 	\\
						& Temperature		& 	\\
						& Solvents			& Solvent A: \ch{H2O}	\\
						& 					& Solvent B: Acetonitrile	\\
						& Method			& Isocratic, 80 \% B \\
						&					& 60 min \\
		MS Parameters	& Capillary Voltage	& 3500 V\\
						& Temperature		& \SI{350}{\celsius}	\\
						& Target Mass		& 250 m/z \\
		\bottomrule
	\end{tabularx}
\end{table}

% subsection mass_spectrometry (end)