\subsection{Chemicals \& Instruments} % (fold)
\label{sec:chemicals_&_instruments}

% section chemicals_&_instruments (end)
\subsection{Strain Cultivation} % (fold)
\label{sec:strain_cultivation}

% section strain_cultivation (end)
\subsection{Sample Preparation} % (fold)
\label{sec:sample_preparation}

% section sample_preparation (end)
\subsection{Chromatographic Methods} % (fold)
\label{sec:chromatographic_methods}

% section chromatographic_methods (end)
\subsubsection{Column Chromatography} % (fold)
\label{sub:column_chromatography}

% subsubsection çolumn_chromatography (end)
\subsubsection{Ion Exchange Chromatography} % (fold)
\label{sub:ion_exchange_chromatography}

% subsubsection ion_exchange_chromatography (end)
\subsubsection{Hydrophilic Interaction Chromatography} % (fold)
\label{ssub:hilic}

Hydrophilic Interaction Chromatography (HILIC) was employed to separate small polar molecules. A   4,6 x 250 mm ZIC-HILIC Column (Merck) was used in combination with an Agilent 1100 HPLC-System.
Milli-Q H2O with 10 mM Ammonium acetate was used as solvent A, while Acetonitrile comprised solvent B. The flow was kept at 0.8 ml/min as to the manufacturers specifications. Isocratic gradients of varying solvent composition and length were used to separate the injected analytes. Detailed method descriptions are available in the appendix.
% subsubsection hilic (end)
\subsubsection{High Performance Liquid Chromatography} % (fold)
\label{sub:hplc}

% subsubsection hplc (end)

\subsubsection{Mass Spectrometry} % (fold)
\label{sub:mass_spectrometry}

A test table should be here

\begin{table}[h]
	\caption{A test table for HPLC Methods}
	\label{tab:asddf}
	\centering
	\begin{tabularx}{\textwidth}{XXX}
		%\toprule
						& \textbf{Component}		& \textbf{Decription}	\\
		\midrule
		HPLC Parameters & System			& 	\\
						& Column			& 	\\
						& Injection volume 	& \SI{50}{\micro\liter}	\\
						& Flow				& 	\\
						& Temperature		& 	\\
						& Solvents			& Solvent A: \ch{H2O}	\\
						& 					& Solvent B: Acetonitrile	\\
						& Method			& Isocratic, 80 \% B \\
						&					& 60 min \\
		MS Parameters	& Capillary Voltage	& 3500 V\\
						& Temperature		& \SI{350}{\celsius}	\\
						& Target Mass		& 250 m/z \\
		\bottomrule
	\end{tabularx}
\end{table}

% subsubsection mass_spectrometry (end)