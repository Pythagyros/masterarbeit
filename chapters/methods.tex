% !TEX root = ../main.tex

\section{Chemicals \& Instruments} % (fold)
\label{sec:chemicals_&_instruments}

	Chemicals and solvents were supplied by Merck, if not specified otherwise. Differing vendors for solvents are listed in Table~\ref{tab:solvent_table}

	\begin{table}
		\caption{Used chemicals and solvents}
		\label{tab:solvent_table}
		\centering
		\begin{tabularx}{\textwidth}{XX}
			\toprule
			\textbf{Chemical}	&	\textbf{Supplier}	\\
			\midrule
			J. T. Baker			&	Acetonitrile		\\
								&	Chloroform			\\
			Alfa Aesar			&	Methyl acetate		\\
			Fisher Chemicals 	&	Ethyl acetate		\\
		\end{tabularx}
	\end{table}

	% The used instruments are listed in in Table~\ref{tab:labins}.

	% \begin{table}[H]
	% 	\caption{Used laboratory Instruments}
	% 	\label{tab:labins}
	% 	\centering
	% 	\begin{tabularx}{\textwidth}{XXX}
	% 		\toprule
	% 		\textbf{Instrument}			& \textbf{Model}		& \textbf{Manufacturer}	\\
	% 		\midrule
	% 		Centrifuges			&	Megafuge 1.0 (R)		&	Heraeus Instruments 	\\
	% 							&	Centrifuge 5417 C 		&	Eppendorf	\\
	% 							&	RC6 Plus Centrifuge 	&	Sorvall	\\
	% 		Lyophilizator		&	LyoVac GT2				&	Leybold \\
	% 		Spectrophotometer	&	BioMate 3S				&	Thermo Fisher \\
	% 		Rotary Evaporator	&	Hei-Vap Precision		&	Heidolph \\
	% 							&	Rotavapor RE + PC 3001 VARIO Pump	&	B\"uchi + vacuubrand \\

	% 		\bottomrule
	% 	\end{tabularx}
	% \end{table}

	High performance liquid chromatography (HPLC) systems were manufactured by Agilent. The components of the HPLC systems are listed in Table~\ref{tab:HPLCtab}.

	\begin{table}[H]
		\caption{Components of HPLC systems}
		\label{tab:HPLCtab}
		\centering
		\begin{tabularx}{\textwidth}{XXX}
			\toprule
							& \textbf{Component}		& \textbf{Description}	\\
			\midrule
			Agilent 1100 Series		&	G1322A		&	Degasser			\\
									&	G1311A		&	Quaternary Pump		\\
									&	G1313A		&	Autosampler			\\
									&	G1316A		&	Column Compartment	\\
									&	G1315B		&	Diode Array Detector	\\
			\midrule
			Agilent 1200 Series		&	G1379B		&	Degasser			\\
									&	G1312A		&	Binary Pump			\\
									&	G1367B		&	Autosampler			\\
									&	G1330B		&	Thermostat			\\
									&	G1316A		&	Column Compartment	\\
									&	G1315B		&	Diode Array Detector	\\
			\midrule
			Agilent 1260 Infinity	&	G4225A		&	Degasser			\\
									&	G1312C		&	Binary Pump			\\
									&	G1329B		&	Autosampler			\\
									&	G1330B		&	Thermostat			\\
									&	G1316A		&	Column Compartment	\\
									&	G1315D		&	Diode Array Detector	\\
			\bottomrule
		\end{tabularx}
	\end{table}

% section chemicals_&_instruments (end)

\clearpage

\section{Strain Cultivation} % (fold)
\label{sec:strain_cultivation}

	\subsection{General cultivation of \emph{Streptomyces} sp. T\"u2401} % (fold)
	\label{sub:streptomyces_sp_t}
		\emph{Streptomyces} sp. T\"u2401 was cultivated in a variety of complex media, which are listed in Table~\ref{tab:media_components}. Agar-plate cultures were grown on ISP2 medium at \SI{29}{\celsius}. \SI{100}{\micro\liter} of spore solution or liquid culture were used for inoculation. Liquid cultures were incubated at \SI{27}{\celsius} in shake flasks with aluminium caps. Pre-cultures were grown in \SI{20}{\milli\liter} of NL 410 in \SI{100}{\milli\liter} flasks and inoculated with plate-grown mycelium. Main cultures were grown in \SI{100}{\milli\liter} medium in \SI{500}{\milli\liter} flasks.
	% subsection streptomyces_sp_t (end)

	\subsection{\emph{Escherichia coli} K12} % (fold)
	\label{sub:escherichia_coli_k12}
		\emph{Escherichia coli} K12 was cultivated in LB medium (\SI{10}{\gram} peptone, \SI{5}{\gram} yeast extract, \SI{10}{\gram} \ch{NaCl} per liter; pre-mixed by Roth) at \SI{37}{\celsius}. Liquid cultures were grown in \SI{10}{\milli\liter} of LB in \SI{100}{\milli\liter} flasks with baffles and spiral and shaken at 200 rpm. Solid cultures were cultivated on plates with 2 \% agar.	
	% subsection escherichia_coli_k12 (end)

	\subsection{Bacillus subtilis 168} % (fold)
	\label{sub:bacillus_subtilis_116}
	
	% subsection bacillus_subtilis_116 (end)

	\subsection{Batch Fermentation} % (fold)
	\label{sub:fermentation}
	The strain T\"u2401 was cultivated at a ten-liter scale in a continuous stirred tank bioreactor. \SI{500}{\milli\liter} of pre-culture were grown in five \SI{500}{\milli\liter} round flasks containing \SI{100}{\milli\liter} of NL 410 medium without \ch{CaCO3}. The pre-cultures were inoculated from stored ISP-agar plates and grown for \SI{72}{\hour} at 27 \si{\celsius}. The pre-cultures were pooled and used to inoculate \SI{9.5}{\liter} of NL OM medium for fermentation. The temperature was kept at \SI{27}{\celsius} with an airflow of \SI{5}{\liter\per\minute} and a rotor speed of 200 rpm. Control samples of \SI{15}{\milli\liter} were taken throughout the process at regular intervals. Fermentation was stopped after \SI{125}{\hour} and the culture broth was harvested. Further processing is described in \ref{sub:processing_of_fermentation_broth}.
	
	% subsection fermentation (end)

	\subsection{Media} % (fold)
	\label{sub:media}
	The used media are listed in Table~\ref{tab:media_components}
	\begin{table}[h]
		\caption[Media components for the cultivation of strain Tü2401]{\textbf{Media components for the cultivation of strain T\"u2401.} All amounts are calculated for one liter of Milli-Q \ch{H2O}. The pH was adjusted with \ch{NaOH} and \ch{HCl}.}
		\label{tab:media_components}
		\centering
		\begin{tabularx}{\textwidth}{XSXS[table-format=3.0]@{\,}s[table-unit-alignment = left]X}
			\toprule
			\textbf{Name} & \textbf{pH}	& \textbf{Component}	& \multicolumn{2}{c}{\textbf{Amount}} & \textbf{Vendor} \\
			\midrule
			ISP2	& 7.3		& Yeast extract 		& 4		& \gram &	Oxoid	\\
					&			& Malt extract 			& 10	& \gram &	Thermo Fisher	\\
			NL 200	& 7,5		& D(-)Mannitol			& 20	& \gram	&	Merck	\\
					&			& Cornsteep Powder		& 20	& \gram	&	Sigma-Aldrich	\\
			\midrule
			NL 300	& 7.5		& D(-)Mannitol			& 20	& \gram	&	Merck	\\
					&			& Cotton Seed			& 20	& \gram	&	Pharmamedia	\\
			\midrule
			NL 410	& 7.0		& Glucose				& 10	& \gram	&	Roth	\\
					&			& Glycerol				& 10	& \gram	&	Acros Organics	\\
					&			& Oatmeal				& 5 	& \gram	&	Holo Bio Hafergold	\\
					&			& Soymeal				& 10	& \gram	&	Hensel	\\
					&			& Yeast extract			& 5 	& \gram	&	Oxoid	\\
					&			& Bacto Casaminoacids	& 5 	& \gram	&	Difco	\\
					&			& \ch{CaCO3}			& 1		& \gram	&			\\
			\midrule
			NL 500	& 8.0		& Starch				& 10	& \gram &	Roth	\\
					&			& Glucose				& 10	& \gram	&	Roth	\\
					&			& Glycerol				& 10	& \gram	&	Acros Organics	\\
					&			& Fish Meal 			& 15	& \gram	&	Sigma-Aldrich	\\
					&			& Sea Salts				& 10	& \gram	&	Sigma-Aldrich	\\
			\midrule	
			OM 		& 7.3		& Oatmeal				& 20	& \gram	&	Holo Bio Hafergold	\\
					&			& Trace metal mix		& 5		& \milli\liter	&\\
			\midrule
			Trace metal mix &	& \ch{CaCl2 * 2 H2O}	& 3		& \gram	&		\\
			 		&			& \ch{Fe^3+} citrate	& 1		& \gram	&		\\
			 		&			& \ch{MnSO4 * H2O}		& 200	& \milli\gram	&\\
			 		&			& \ch{ZnCl2}			& 100	& \milli\gram	&\\
			 		&			& \ch{CuSO4 * 5 H2O}	& 25	& \milli\gram	&\\
			 		&			& \ch{Na2B4O7 * 10 H2O}	& 20	& \milli\gram	&\\
			 		&			& \ch{CoCl2 * 6 H2O}	& 4		& \milli\gram	&\\
			 		&			& \ch{Na2MoO4 * 2 H2O}	& 10	& \milli\gram	&\\
			\bottomrule
		\end{tabularx}
	\end{table}
	% subsection media (end)
% section strain_cultivation (end)

\clearpage

\section{Sample Preparation} % (fold)
\label{sec:sample_preparation}

	\subsection{Processing of Fermentation Broth} % (fold)
	\label{sub:processing_of_fermentation_broth}
	The harvested fermentation broth was supplemented with diatomaceous earth and filtered through Pall T 1500 filter plates (relative retention range 10 - 30 µm). The remaining filter cake was discarded and the filtrate transferred to a stirring bucket % englischen begriff nachsehen
	Two liters of ethyl acetate were added to the filtrate and stirred for 30 min. After completed phase-separation, the organic phase was collected and the aqueous phase reused for further extraction. The process was repeated five times.
	% subsection processing_of_fermentation_broth (end)

% section sample_preparation (end)

\clearpage

\section{Chromatographic Methods} % (fold)
\label{sec:chromatographic_methods}
% All Temperatures set at 25 °C
% section chromatographic_methods (end)

	\subsection{Thin Layer Chromatography} % (fold)
	\label{sub:thin_layer_chromatography}
	Thin layer Chromatography was performed with reverse extracts of T\"u2401 on TLC Silica Gel 60 F\textsubscript{254} plates by Merck.
	Aqueous samples were applied by pipetting \SI{1}{\micro\liter} at a time and letting the plate dry until the next application. The TLC chambers were filled up to \SI{1}{\centi\meter} with solvent and incubated for \SI{12}{\hour}. The plates were run until either 75 \% of the plate had been soaked or \SI{2}{\hour} had passed. The solvents used as mobile phases are listed in Table~\ref{tab:tlc_solvents}.

	\begin{table}[h]
		\caption{Mobile phases used for Thin-Layer Chromatography}
		\label{tab:tlc_solvents}
		\centering
		\begin{tabularx}{\textwidth}{XX}
			\toprule
			\textbf{Solvent}			& \textbf{Composition}	\\
			\midrule
			Acetonitrile / Water				& 85:15		\\
			Butanol / Acetic acid / Water		& 14:3:2	\\
												& 42:10:7	\\
			Butanol / Ethanol / Water			& 3:2:1		\\
			Ethyl acetate / 2-Propanol / Water	& 6:3:1		\\
			Chloroform / Methanol				& 8:3		\\
			\bottomrule
		\end{tabularx}
	\end{table}

	The working orcin staining solution was prepared by mixing two storage solutions, solution A and B, at a ratio of 10:1 (v/v). Solution A contained 1 \% (w/v) \ch{Fe^{III}Cl} in 10 \% sulfuric acid, solution B contained 6 \% (w/v) Orcin in ethanol. The plates were sprayed with the working solution and treated with a heat gun for a few seconds.

	Preparative samples were obtained by scraping the silica off the unstained plate and collecting it in reaction tubes. The samples were then extracted with \SI{1}{\milli\liter} methanol, vortexed and sonicated for \SI{30}{\minute}. After centrifugation at 14,000 rpm for \SI{5}{\minute}, the supernatant was transferred to a new tube and the extraction process repeated. The methanolic samples were dried at \SI{30}{\celsius} and resuspended in an amount of water equal to the sample initially applied on the TLC plate.
	% subsection thin_layer_chromatography (end)

	\subsection{Ion Exchange Chromatography} % (fold)
	\label{sub:ion_exchange_chromatography}

		% \begin{table}[h]
		% 	\caption{Ion exchange materials}
		% 	\label{tab:ion_exchangers}
		% 	\centering
		% 	\begin{tabularx}{\textwidth}{XXXX}
		% 		\toprule
		% 		\textbf{Name}	& \textbf{Matrix}	& \textbf{Functional groups}	& \textbf{Ionic Form}	\\
		% 		\midrule
		% 		Diaion\textsuperscript{\texttrademark} SA11A	&	Polystyrene Divinylbenzene	&	Trimethyl ammonium 	&	\ch{Cl^-}	\\
		% 		Dowex\textsuperscript{\texttrademark} 50WX4	&&& \\
		% 		\bottomrule
		% 	\end{tabularx}
		% \end{table}
	% subsection ion_exchange_chromatography (end)

	\subsection{Hydrophilic Interaction Chromatography} % (fold)
	\label{sub:hilic}

	Hydrophilic Interaction Chromatography (HILIC) with performed with a 4,6 x 250 mm ZIC-HILIC Column (Merck). It features zwitterionic, functional groups on poly(etherether ketone) (PEEK) material.
	10 mM Ammonium acetate in Milli-Q \ch{H2O} was used as solvent A, while Acetonitrile comprised solvent B. Detailed method descriptions regarding solvent composition, flow and duration are listed in the appendix.
	% subsection hilic (end)

	\subsection{High Performance Liquid Chromatography} % (fold)
	\label{sub:hplc}
	% subsection hplc (end)

	\subsection{Mass Spectrometry} % (fold)
	\label{sub:mass_spectrometry}

A test table should be here

\begin{table}[h]
	\caption{A test table for HPLC Methods}
	\label{tab:asddf}
	\centering
	\begin{tabularx}{\textwidth}{XXX}
		%\toprule
						& \textbf{Component}		& \textbf{Description}	\\
		\midrule
		HPLC Parameters & System			& 	\\
						& Column			& 	\\
						& Injection volume 	& \SI{50}{\micro\liter}	\\
						& Flow				& 	\\
						& Temperature		& 	\\
						& Solvents			& Solvent A: \ch{H2O}	\\
						& 					& Solvent B: Acetonitrile	\\
						& Method			& Isocratic, 80 \% B \\
						&					& 60 min \\
		MS Parameters	& Capillary Voltage	& 3500 V\\
						& Temperature		& \SI{350}{\celsius}	\\
						& Target Mass		& 250 m/z \\
		\bottomrule
	\end{tabularx}
\end{table}

% subsection mass_spectrometry (end)