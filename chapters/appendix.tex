% !TEX root = ../main.tex

\section{HPLC Methods} % (fold)
\label{sec:hplc_methods}
	Vielleicht wenn hier text steht
	\begin{table}[htbp]
		\caption{The standard HILIC method}
		\label{tab:hilic_standard}
		\centering
		\begin{tabularx}{\textwidth}{XXX}
			\toprule
						& 	\textbf{Parameter}	& \textbf{Value}	\\
			\midrule
			Column		& Name			& Merck SeQuant\textsuperscript{\textregistered} ZIC\textsuperscript{\textregistered}-HILIC 	\\
						& Dimensions	& 150 $\times$ \SI{4.6}{\milli\meter}	\\
						& Stat. phase	& PEEK \SI{3.5}{\micro\meter} \SI{200}{\angstrom} \\
						& Injection volume& \SI{50}{\micro\liter}	\\
						& Temperature	& \SI{25}{\celsius}		\\
						& Flow			& \SI{0.8}{\milli\liter\per\minute}	\\
						& Method 		& Isocratic 80~\%	\\
						&				& for \SI{45}{\minute}			\\

			\bottomrule
		\end{tabularx}
	\end{table}

\begin{table}[h]
	\caption{A test table for HPLC Methods}
	\label{tab:asddf}
	\centering
	\begin{tabularx}{\textwidth}{XXX}
		%\toprule
						& \textbf{Component}		& \textbf{Description}	\\
		\midrule
		HPLC Parameters & System			& 	\\
						& Column			& 	\\
						& Injection volume 	& \SI{50}{\micro\liter}	\\
						& Flow				& 	\\
						& Temperature		& 	\\
						& Solvents			& Solvent A: \ch{H2O}	\\
						& 					& Solvent B: Acetonitrile	\\
						& Method			& Isocratic, 80 \% B \\
						&					& 60 min \\
		MS Parameters	& Capillary Voltage	& 3500 V\\
						& Temperature		& \SI{350}{\celsius}	\\
						& Target Mass		& 250 m/z \\
		\bottomrule
	\end{tabularx}
\end{table}
% section hplc_methods (end)