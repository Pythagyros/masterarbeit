% !TEX root = ../main.tex

\section{Determination of Extraction Conditions} % (fold)
\label{sec:determination_of_extraction_conditions}

% section determination_of_extraction_conditions (end)

\section{Chromatographic Separation} % (fold)
\label{sec:chromatographic_separation}

    \subsection{Reverse-Phase HPLC} % (fold)
    \label{sub:reverse_phase_hplc}

    % subsection reverse_phase_hplc (end)

    \subsection{HPLC with Amino Column} % (fold)
    \label{sub:hplc_with_amino_column}

    % subsection hplc_with_amino_column (end)

    \subsection{Hydrophilic Interaction Chromatography} % (fold)
    \label{sub:hydrophilic_interaction_chromatography}

    % subsection hydrophilic_interaction_chromatography (end)

    \subsection{Ion exchange Chromatography} % (fold)
    \label{sub:ion_exchange_chromatography}

    % subsection ion_exchange_chromatography (end)

    \subsection{Thin-Layer Chromatography} % (fold)
    \label{sub:thin_layer_chromatography}

    % subsection thin_layer_chromatography (end)

% section chromatographic_separation (end)

\section{Dereplication} % (fold)
\label{sec:dereplication}

    \subsection{HPLC Mass Spectrometry} % (fold)
    \label{sub:hplc_mass_spectrometry}

    % subsection hplc_mass_spectrometry (end)

    \subsection{Trimethylsilane Derivatization and Gas Chromatography} % (fold)
    \label{sub:trimethylsilane_derivatization_and_gas_chromatography_results}

    % subsection trimethylsilane_derivatization_and_gas_chromatography (end)

% section dereplication (end)

\section{Antibacterial Activity Spectrum} % (fold)
\label{sec:antibacterial_activity_spectrum}


    \subsection{Activity against E. coli} % (fold)
    \label{sub:activity_against_e_coli}

    % subsection activity_against_e_coli (end)

    \subsection{Activity against B. subtilis} % (fold)
    \label{sub:activity_against_b_subtilis}

    % subsection activity_against_b_subtilis (end)

    \subsection{Extraction of yorB-inducing Compound} % (fold)
    \label{sub:extraction_of_yorb_inducing_compound}

    Tü2401 displayed positive results in the yorB-Assay \todo{Referenz einsetzen} when grown on ISP2 plates \todo{Referenz einsetzen}. However, all previously generated samples from liquid cultures only showed antibacterial activity.
    %%%%%%%%%%%%%%
    %Einsetzen:
    %%%%%%%%%%%%%%
    % Absatz über Differenzierung von Streptomyceten
    % Einfluss auf Sekundärmetabolismus
    % Überleitung zu Schlussfolgerung: Evtl zweiter Compound nur auf agar produziert
    %%%%%%%%%%%%%%%
    Three cultivation strategies were developed to induce production of the putative compound 2 (PC2):

    \begin{enumerate}
        \item Standing cultures could allow the formation of aerial mycelium at the medium surface. Thus, enabling the synthesis of PC2 and allowing extraction of the liquid medium. Two \SI{500}{\milli\liter} flasks were each filled with \SI{100}{\milli\liter} of liquid ISP2 medium and inoculated with \SI{1}{\milli\liter} of a one-week old NL~410 shake culture. One flask was sealed with an ordinary aluminium cap (AC), the other one with an air-permeable foam cap (FC). Both flasks were cultivated for four days, before the medium was centrifuged and the supernatant filtrated. \SI{50}{\milli\liter} aliquots of each flask were extracted with either BuOH or EtAc, and both phases were collected seperately. The organic phases were dried at \SI{40}{\celsius} and solved in \SI{1}{\milli\liter} MeOH. Both phases of each flask were subjected to the yorB-Assay.
        \item ISP2 agar plates were previously known to enable synthesis of PC2 and could be extracted with prior breakup. Ten round ISP2 agar plates were inoculated with \SI{100}{\micro\liter} of a one-week old NL~410 shake culture and incubated for four days. One half of the plates was extracted with BuOH, the other half was extracted with EtAc.
        \item ISP2 agar plates could also be prepared with low-melting-point agarose (LMPA). This would allow melting of the plates at lower temperatures. Thus, allowing for easier extraction with reduced risk of thermal decomposition during the process. Six ISP2 LMPA agar plates were inoculated with \SI{100}{\micro\liter} of a week-old NL~410 shake culture and incubated for four days. The plates were then melted at \SI{70}{\celsius} and extracted with BuOH and EtAc.
    \end{enumerate}

    All generated extracts and their respective aqueous phases were tested for bioactivity. The standard bioassay was used to determine the antibiotic activity against \textit{E. coli} K12 and \textit{B. subtilis} 168, and the yorB-Assay was used to test for the mode of action. Additonally, agar stamps of grown ISP2 and ISP2 LMPA plates were tested in the yorB-Assay. A summary of the results is shown in \todo{Tabelle einsetzen}.

    \begin{table}[htbp]
        \caption[Bioassay results from agar-plate and standing culture extraction]{\textbf{Bioassay results from agar-plate and standing culture extraction}.\\
        Samples were screened for antibiotic activity against \textit{E. coli} K12 and \textit{B. subtilis} 168 and for promotor induction in the yorB-Assay.
        Samples from standing cultures with foam cap (FC) or aluminium cap (AC), ISP2 agar plates (ISP2) and ISP2 agar plates with low-melting-point agarose (LMPA). Samples were extracted with butanol (BuOH) or ethyl acetate (EtAc) and tested alongside their respective aqueous phases (aq.). \emph{Legend}: \textbf{-} No activity; \textbf{+ / ++ / +++} antibiotic activity with inhibition zone of 1~/~1.0-1.5~/~>1.5~cm; \textbf{n.e.} result non-evaluable}
        \label{tab:yorB_assay_results}
        \centering
        \begin{tabularx}{\textwidth}{XXXXX}
            \toprule
            & \multicolumn{3}{c}{Antibacterial} & positive \\
            \cline{2-4}
            \textbf{Sample} & \textbf{\textit{E. coli}}     & \textbf{\textit{B. subtilis}}  & \textbf{yorB}  & \textbf{yorB}    \\
            \midrule
            FC BuOH         & -     & -     & -     & -    \\
            FC BuOH aq.     & -     & -     & -     & -    \\
            FC EtAc         & -     & +++   & ++    & -    \\
            FC EtAc aq.     & -     & -     & -     & -    \\
            AC BuOH         & -     & -     & -     & -    \\
            AC BuOH aq.     & -     & -     & -     & -    \\
            AC EtAc         & -     & -     & +     & -    \\
            AC EtAc aq.     & -     & -     & -     & -    \\
            ISP2 BuOH       & -     & n.e.  & +++   & -    \\
            ISP2 BuOH aq.   & -     & +     & -     & -    \\
            ISP2 EtAc       & -     & ++    & -     & -    \\
            ISP2 EtAc aq.   & +     & n.e.  & -     & +    \\
            ISP2 plaque     &       &       & ++    & ++   \\
            LMPA BuOH       & +     & n.e.  & +++   & -    \\
            LMPA BuOH aq.   & -     & n.e.  & -     & -    \\
            LMPA EtAc       & -     & n.e.  & +++   & -    \\
            LMPA EtAc aq.   & -     & n.e.  & -     & -    \\
            LMPA plaque     &       &       & ++    & ++   \\
            \bottomrule
        \end{tabularx}
    \end{table}
    % subsection extraction_of_yorb_inducing_compound (end)

% section antibacterial_activity_spectrum (end)

\section{Genomic Analysis} % (fold)
\label{sec:genomic_analysis}

    \subsection{Phylogeny of Strain Tü2401} % (fold)
    \label{sub:phylogeny_of_strain_tue2401}

    % subsection phylogeny_of_strain_tü2401 (end)

    \subsection{AntiSMASH Cluster Analysis} % (fold)
    \label{sub:antismash_cluster_analysis}

    % subsection antismash_cluster_analysis (end)

% section genomic_analysis (end)

