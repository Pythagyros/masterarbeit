%%% SETUP DOCUMENT CLASS

\documentclass[%
	paper=a4,
	fontsize=12pt,
	%BCOR=5mm,																										% Korrektur Binderand
	oneside,
	openany,									%chapterstart
	headinclude,
	fontinclude=true,
	cleardoublepage=empty,
	pagesize=pdftex,
	draft=false,	
	%headsepline,
	%footsepline,
	%toc=bib, index=totoc,
	%toc=listof,
	%report,
	english]{scrbook}																								% http://www.golatex.de/wiki/KOMA-Script

\usepackage{scrhack}																								% http://www.golatex.de/wiki/KOMA-Script

%%% SETUP DOCUMENT LANGUAGE

%\usepackage{ngerman} 																								% http://texdoc.net/texmf-dist/doc/generic/german/gerdoc.pdf
\usepackage[english]{babel}																							% http://www.ctan.org/tex-archive/language/babel



%%% SETUP DOCUMENT & FONT ENCODING

\usepackage[utf8x]{inputenx}															% http://www.ctan.org/pkg/inputenx
\usepackage{ucs}																		% http://ctan.org/pkg/ucs

\usepackage[T1]{fontenc}																% http://ctan.org/pkg/fontenc
\usepackage[singlelinecheck=false]{caption}												%singlelinecheck to align table caption

%%% LOAD MATH FONTS

\usepackage{amsmath}																	% http://ctan.org/pkg/amsmath
\usepackage{amsfonts}																	% http://ctan.org/pkg/amsfonts
\usepackage{amssymb}																	% http://milde.users.sourceforge.net/LUCR/Math/mathpackages/amssymb-symbols.pdf
\usepackage{amsthm}																		% http://ctan.org/pkg/amsthm
\usepackage{dsfont}																		% http://www.ctan.org/tex-archive/fonts/doublestroke

%%% LOAD TEXT FONTS

\usepackage{lmodern}																								% http://www.ctan.org/tex-archive/fonts/lm/
\usepackage[scaled=.85]{luximono}																					% http://www.ctan.org/pkg/luximono
\usepackage[style=swiss]{csquotes}																					% http://www.ctan.org/pkg/csquotes
\usepackage{microtype}																								% http://www.ctan.org/pkg/microtype
%\usepackage[style=swiss]{csquotes-de}																				% http://www.ctan.org/pkg/csquotes-de
%\usepackage{microtype-de}																							% http://www.ctan.org/pkg/microtype

%%% SETUP GRAPHICS

\usepackage{wrapfig}																								% http://www.ctan.org/pkg/wrapfig
\usepackage{subcaption}																						% 2 Bilder nebeneinander
\usepackage[pdftex]{graphicx}																						% http://www.ctan.org/pkg/graphicx
	\DeclareGraphicsExtensions{.pdf,.png,.jpg}
	\graphicspath{{./images/}}
																		%Abb. statt ABbildung in Caption

%%% SETUP DOCUMENT STRUCTURE

%%Set uo argins / Rand of Document %%

\usepackage[inner=0.8in, outer=0.8in, top=1in, bottom=1in, bindingoffset=0mm, a4paper]{geometry}										% http://www.ctan.org/pkg/geometry
%\usepackage[left=20mm, right=20mm, top=20mm, bottom=20mm, a4paper]{geometry-de}									% http://www.ctan.org/pkg/geometry	
\usepackage{rotating}																								% http://www.ctan.org/pkg/rotating
%\usepackage{landscape}																								% http://www.ctan.org/pkg/landscape
\usepackage{setspace}																								% http://www.ctan.org/pkg/setspace
\onehalfspacing
\setlength\parindent{0pt}																							% global disable indentation

%%% DEFINE CUSTOM COMMANDS

% place here ...

%%% SETUP DOCUMENT COUNTERS

\usepackage{chngcntr}																								% http://www.ctan.org/pkg/chngcntr
	\counterwithout{footnote}{chapter}																				% don't use chapter numbering in footnotes

\setcounter{secnumdepth}{4}																							% depth of sections (i.e. 4 => 1.1.1.1) in document
\setcounter{tocdepth}{4}																							% depth of sections (i.e. 4 => 1.1.1.1) in table of contents

%%% SETUP DOCUMENT PAGE STYLE

%\usepackage{fancyhdr}
%	%\pagestyle{fancy}																								% select page style
%	\fancyhf{}																										% clear definitions
	
%	\fancyhead[EL]{}																								% header even left 
%	\fancyhead[OL]{}																								% header odd left
%	\fancyhead[EC]{}																								% header even center
%	\fancyhead[OC]{}																								% header odd center
%	\fancyhead[ER]{}																								% header even right
%	\fancyhead[OR]{}																								% header odd right
	
%	\fancyfoot[EL]{}																								% footer even left
%	\fancyfoot[OL]{}																								% footer odd left
%	\fancyfoot[EC]{}																								% footer even center
%	\fancyfoot[OC]{}																								% footer odd center
%	\fancyfoot[ER]{}																								% footer even right
%	\fancyfoot[OR]{}																								% footer odd right

%%% SETUP DOCUMENT BIBLIOGRAPHY

%\usepackage[super,numbers]{natbib}																									% http://www.ctan.org/pkg/natbib
\usepackage{achemso}
%\setkeys{acs}{biochem}
\setkeys{acs}{articletitle = true}
 %should work with chemso package, but doesn't
%\bibliographystyle{biochem}


																						% change your bibstyle here ametsoc

\usepackage[fixlanguage]{babelbib}																					% http://www.ctan.org/pkg/babelbib
	\selectbiblanguage{english}																						% use german bibliography translations

%%% SETUP DOCUMENT GLOSSARY

\usepackage[nonumberlist,toc]{glossaries}																			% http://www.ctan.org/pkg/glossaries
	\makeglossaries																									% generates glossaries

%%% SETUP DOCUMENT INDEX

\usepackage{makeidx}																								% http://www.ctan.org/pkg/makeidx
	\makeindex																										% generates index

%%% SETUP DOCUMENT TABLES

\usepackage{multirow}																								% http://www.ctan.org/pkg/multirow
\usepackage{tabularx}																								% http://www.ctan.org/pkg/tabularx
\usepackage{booktabs}																								% http://www.ctan.org/pkg/booktabs
%\usepackage{booktabs-de}																							% http://www.ctan.org/pkg/booktabs-de
\usepackage{array}																									% http://www.ctan.org/pkg/array
%\usepackage{threeparttable}

%\usepackage{floatrow}																								% tabelle bild nebeneinander
\newcolumntype{R}{>{\raggedright\arraybackslash}X}%														% ctabel/tabularx rechts ausgerichtet zellen
\newcolumntype{L}{>{\raggedleft\arraybackslash}X}%

%%% SETUP SOURCE CODE LISTINGS

\usepackage{listings}																								% http://www.ctan.org/tex-archive/macros/latex/contrib/listings/
	\lstset{%
		language=C,                	  																				% the language of the code
 		basicstyle=\footnotesize,       																			% the size of the fonts that are used for the code
		numbers=left,                   																			% where to put the line-numbers
		numberstyle=\tiny\color{gray}, 		 																		% the style that is used for the line-numbers
		stepnumber=2,                   																			% the step between two line-numbers. If it's 1, each line will be numbered
		numbersep=5pt,                 		 																		% how far the line-numbers are from the code
		backgroundcolor=\color{white},  																			% choose the background color. You must add \usepackage{color}
		showspaces=false,               																			% show spaces adding particular underscores
		showstringspaces=false,       		  																		% underline spaces within strings
		showtabs=false,                 																			% show tabs within strings adding particular underscores
		frame=single,                   																			% adds a frame around the code
		rulecolor=\color{black},	        																		% if not set, the frame-color may be changed on line-breaks 
		tabsize=2,                  	    																		% sets default tabsize to 2 spaces
		captionpos=b,                   																			% sets the caption-position to bottom
		breaklines=true,                																			% sets automatic line breaking
		breakatwhitespace=false,        																			% sets if automatic breaks should only happen at whitespace
		title=\lstname,		                 																		% show the filename of files included with \lstinputlisting;
																													% also try caption instead of title
		keywordstyle=\color{red},		      		 																% keyword style
		commentstyle=\color{red},       																			% comment style
		stringstyle=\color{red},        																			% string literal style
		escapeinside={\%*}{*)},         																			% if you want to add a comment within your code
		morekeywords={*,...}            																			% if you want to add more keywords to the set		
	}

%%% SETUP DOCUMENT COLORS

%\usepackage[usenames,dvipsnames]{xcolor}																			% http://www.ctan.org/pkg/xcolor
	%\definecolor{myOrange}{RGB}{255,127,0}																			% set up custom colors using RGB values

%%% SETUP KOMASCRIPT SETTINGS

\setcapindent{1em}																									% set image caption line break indentation

\setkomafont{sectioning}{\normalfont\bfseries}     																	% sectioning titles with normal font
\setkomafont{captionlabel}{\normalfont\bfseries}   																	% bold caption titles
\setkomafont{pagehead}{\normalfont\itshape}         																% italic page title
\setkomafont{descriptionlabel}{\normalfont\bfseries}																% bold description titles

%%% SETUP CHEMICAL DRAWINGS / FORMULA

\usepackage[version=4]{mhchem}																						% http://www.ctan.org/pkg/mhchem
%\usepackage{rsphrase}																								% contained in mhchem 
%\usepackage{chemexec}																								% http://www.ctan.org/pkg/chemexec
%\usepackage{stree} 																									% http://www.ctan.org/pkg/streetex 	!! stree and ochem are not compatible
%\usepackage{ochem}																									% http://www.ctan.org/pkg/ochem		!! stree and ochem are not compatible
%\usepackage{bpchem}																									% http://www.ctan.org/pkg/bpchem
%\usepackage{chembst}																								% http://www.ctan.org/pkg/chembst
%usepackage{chemcompounds}																							% http://www.ctan.org/pkg/chemcompounds
%\usepackage{chemcono}																								% http://www.ctan.org/pkg/chemcono
%\usepackage{chem-journal}																							% http://www.ctan.org/pkg/chem-journal
\usepackage{chemfig}																								% http://www.ctan.org/pkg/chemfig
%\usepackage{chemmacros}
%\usechemmodule{all}																							% z.B. NMR siehe documentation pdf

%%% SETUP DOCUMENT INTERACTION

\usepackage{url}																									% http://www.ctan.org/tex-archive/macros/latex/contrib/url/
	\urlstyle{same}																									% use same style as document instead of mono-spaced font
	
\usepackage{pdfpages}																								% http://www.ctan.org/tex-archive/macros/latex/contrib/pdfpages/

% \usepackage[ %
% 	%pagebackref,																									% enable back referencing to bibliography
% 	bookmarks,
% 	pdfpagelabels,
% 	plainpages=true	
% ]{hyperref}																											% http://www.ctan.org/tex-archive/macros/latex/contrib/hyperref/
% 	\hypersetup{ %
% 		unicode=true,																								% allows to use Unicode in Acrobat’s bookmarks
% 		pdftoolbar=true,																							% show or hide Acrobat’s toolbar
% 		pdfmenubar=true,																							% show or hide Acrobat’s menu
% 		pdffitwindow=true,																							% resize document window to fit document size
% 		pdfstartview={FitH, FitV}																					% fit the width of the page to the window
% 		pdftitle={TITLE},																							% define the title in the "Document Info" window of Acrobat
% 		pdfauthor={AUTHOR},																							% the name of the PDF’s author, like above
% 		pdfsubject={SUBJECT},																						% subject of the document,like above
% 		pdfcreator={LaTeX with hyperref},																			% creator of the document, like above
% 		pdfproducer={pdfTeX, Version 3.1415926-2.4-1.40.13},														% producer of the document, like above
% 		pdfkeywords={KEY1, KEY2, ...},																				% list of keywords, separated by brackets
% 		pdfnewwindow=true,																							% define if a new window should get opened on external links
% 		breaklinks=true,																							% allow proper word wrapping for long captions in toc, listoffigures etc.
% 		colorlinks=true,																							% enable link colors
% 		linkcolor=black, 																								% color of internal links (sections, pages, etc.)
% 		citecolor=black, 																							% color of citation links (bibliography)
% 		filecolor=magenta, 																							% color of file links
% 		urlcolor=cyan, 																								% color of URL links (mail, web)
% 		linktoc=all,																								% defines which part of an entry in toc is made into a link
% 		pdflang=de,																									% define document language
% 		final
% 	}	

	% MAth Symbole ?

	\usepackage{relsize}

	%footnote 																										blank footnotes
	
	\usepackage{textcomp}																							% griechische buchstaben nicht kursiv
\usepackage{afterpage}

	%\usepackage{float}