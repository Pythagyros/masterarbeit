% Author: Rasmus Pank Roulund
\documentclass[options]{standalone}
% \usepackage{tikz}
% \usetikzlibrary{shapes,arrows}

% \newcommand*{\h}{\hspace{5pt}}% for indentation
% \newcommand*{\hh}{\h\h}% double indentation
\begin{document}
  % setting the typeface to sans serif and the font size to small
  % the scope local to the environment
  \sffamily
  \footnotesize
  \begin{tikzpicture}[auto,
    %decision/.style={diamond, draw=black, thick, fill=white,
    %text width=8em, text badly centered,
    %inner sep=1pt, font=\sffamily\small},
    block_center/.style ={rectangle, draw=black, thick, fill=white,
      text width=10em, text centered,
      minimum height=2em},
    block_left/.style ={rectangle, draw=black, thick, fill=white,
      text width=16em, text ragged, minimum height=4em, inner sep=6pt},
    block_noborder/.style ={rectangle, draw=none, thick, fill=none,
      text width=10em, text centered, minimum height=1em},
    block_assign/.style ={rectangle, draw=black, thick, fill=white,
      text width=18em, text ragged, minimum height=3em, inner sep=6pt},
    block_lost/.style ={rectangle, draw=black, thick, fill=white,
      text width=16em, text ragged, minimum height=3em, inner sep=6pt},
      line/.style ={draw, thick, -latex', shorten >=0pt}]
    % outlining the flowchart using the PGF/TikZ matrix funtion
    \matrix [column sep=5mm,row sep=10mm] (m) {
    & \node [block_center] (1) {\textbf{10 L Culture Broth}}; & \\
    % \node [block_noborder] (15) {active against K12};\\
    %& & \node [block_noborder] (15) {Filtrated through Pall T 1500}; \\
    & \node [block_center] (2) {\textbf{8.5 L Filtrate}}; & \\
    \node [block_center] (3) {\textbf{8.5 L Extract}}; &
    \node [block_noborder] (35) {Evaporation and Lyophilization}; & 
    \node [block_center] (4) {\textbf{8 L Reverse Extract}}; \\
    \node [block_center] (5) {\textbf{235 g Extract}}; &
    \node [block_noborder] (55) {Final Weight}; & 
    \node [block_center] (6) {\textbf{91 g Reverse Extract}}; \\
    % \node [block_noborder] (7) {inactive against \emph{B.~subtilis} and K12}; & &
    % \node [block_noborder] (8) {active against \emph{B.~subtilis} and K12}; \\
  };
  \begin{scope}[every path/.style=line]
  \path (1) -- (2);
  \path (2) -- (3);
  \path (2) -- (4);
  \path (3) -- (5);
  \path (4) -- (6);
  \end{scope}
  \end{tikzpicture}
\end{document}
